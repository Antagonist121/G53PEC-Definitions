\documentclass[]{../../DefinitionFormat}

\begin{document}

\begin{definition}{Society}
	\begin{summary}
	"A society is an association of people organized under a system of rules designed to advance the good of its members over time." - Addison Wesley, Ethics for the Information Age
	\end{summary}
	
	\par People in a society will still compete with one another over limited benefits. A society will have rules describing can and cannot be done in these various situations called morality.
\end{definition}

\begin{definition}{Pervasive nature of computing}
	A paradigm also known as ubiquitous computing describing a large network of small distributed computers that give the appearance of being everywhere.
	
	\par As computers in society become more ubiquitous, the subject of ethics and professional approaches to innovating devices becomes increasingly important.
\end{definition}

\begin{definition}{Ethics}
	\begin{summary}
	"Ethics is the philosophical study of morality, a rational examination into people’s
	moral beliefs and behaviour." - Addison Wesley, Ethics for the Information Age
	\end{summary}
	
	\par Ethics involves observing and evaluating actions in a society to determine if it falls within the set moral guidelines. This allows the observer to judge the action as moral or immoral.
\end{definition}

\begin{definition}{Realm of Ethics}
	\par Ethics focuses on voluntary, moral choices that people make when presented with multiple options. Actions caused from involuntary choices or reflex actions are outside the realm of ethics. 
	\par If the involuntary choice or reflex action was a consequence of a voluntary immoral choice previously then it would be within the realm of ethics.
\end{definition}

\begin{definition}{Ethical Theory}
	\par An ethical theory is a framework for moral decision making.
\end{definition}

\begin{definition}{Relativism}
	\begin{summary}
		"Relativism is the theory that there are no universal moral norms of right and wrong.
		According to this theory, different individuals or groups of people can have completely
		opposite views of a moral problem, and both can be right." - Addison Wesley, Ethics for the Information Age
	\end{summary}
	\par There are 2 types of relativism; subjective relativism and cultural relativism.
\end{definition}

\begin{definition}{Subjective Relativism}
	\par Subjective Relativism states that each person has a different stance on what is morally right and wrong.
	
	\renewcommand{\labelitemi}{$+$}
	\begin{itemize}
		\item Rational and intelligent people can have completely opposite opinions on moral issues.
		\item Ethical debates are disagreeable and pointless. Who is to say which side of an issue is correct?
	\end{itemize}
	\renewcommand{\labelitemi}{$-$}
	\begin{itemize}
		\item Allows people to rationalize bad behaviour because "who is the opposition to say that it's wrong?".
		\item Subjective Relativism gives us no framework for comparing the morality of 2 individual's actions.
		\item Individuals deciding what is right and wrong allows them to reach any conclusion they see fit.
	\end{itemize}
\end{definition}

\begin{definition}{Cultural Relativism}
	\begin{summary}
		"Cultural relativism is the ethical theory that the meaning of “right” and “wrong”
		rests with a society’s actual moral guidelines. These guidelines vary from place to place
		and from time to time." - Addison Wesley, Ethics for the Information Age
	\end{summary}
	
	\par Cultural relativism implies that there is no \textit{universal human morality}.
	
	\renewcommand{\labelitemi}{$+$}
	\begin{itemize}
		\item Different social contexts demand different moral guidelines.
		\item It is arrogant for different societies (time and place) to judge each other.
	\end{itemize}
	\renewcommand{\labelitemi}{$-$}
	\begin{itemize}
		\item Cultural relativism prevents one society from speaking out against the (perceived) immoral actions of another society.
		\item Newcomers to a society have no way of determining the moral guidelines of a particular society.
		\item Cultural relativism does not help when the society is split on the issue.
		\item No provided way to reconcile between 2 societies.
	\end{itemize}
\end{definition}

\begin{definition}{Equivalence Fallacy}
	Trying to equate 2 related but distinct things. For example, the statement "God is Good". Is an action good because God commands it or does God command it because it is good?
\end{definition}

\begin{definition}{Divine Command Theory}
	\begin{summary}
		"The divine command theory is based on the idea that good actions are those aligned
		with the will of God and bad actions are those contrary to the will of God. Since the holy
		books contain God’s directions, we can use the holy books as moral decision-making
		guides." - Addison Wesley, Ethics for the Information Age
	\end{summary}
	
	\par Not all in a religion subscribe to divine command theory, most sects in religions develop a moral code with both their holy scriptures and other sources.
	
	\renewcommand{\labelitemi}{$+$}
	\begin{itemize}
		\item God is all-good and all-knowing and therefore his will should be aligned with to ensure that as many people as possible are happy.
		\item If our goal is to create a society where everyone obeys the same moral laws then we should use God as the ultimate authority.
	\end{itemize}
	\renewcommand{\labelitemi}{$-$}
	\begin{itemize}
		\item There are many holy books with conflicting moral codes.
		\item A multi-cultural society will never adopt a religion-based morality.
		\item Not all moral problems are addressed in holy books.
		\item "God is good" is an \textit{equivalence fallacy}.
	\end{itemize}
	
	\par Moral guidelines from divine command theory are not a result of logical progression from underlying principles and is therefore not a powerful weapon for ethical debate in a secular society. \textbf{Divine Command Theory is not a workable theory}.
\end{definition}

\end{document}
