\documentclass[]{../../DefinitionFormat}

\begin{document}
\begin{definition}{Critical Thinking}
	\begin{summary}
	The ability to think clearly and rationally about what to do or believe.
	\end{summary}
	
	\par The ability to engage in reflective and independent thinking. This means to understand connections between ideas, identify construct and evaluate arguments, and detect mistakes in your reasoning. It is also about reflecting of your own beliefs and values.
\end{definition}

\begin{definition}{Rationality}
	\begin{summary}
		The state of being reasonable based on facts or logic
	\end{summary}
	
	\par Our beliefs are aligned with our reasons to believe, and our actions reflect our reasons for action. There are rules that you cannot argue against, but the language of those rules can be open to interpretation and introduce ambiguities.
\end{definition}

\begin{definition}{Logic}
	\begin{summary}
		The principles of correct reasoning.
	\end{summary}
	
	\par There are two questions related to logic, how we get to a certain conclusion and how we justify decisions. In order to look at this, we focus on the principles that govern the validity of arguments.
\end{definition}

\begin{definition}{Assertions}
	
	\par A statement, "Shakespeare wrote the play Hamlet", "G53PEC is a pile of shite", etc.
\end{definition}


\begin{definition}{Argument}
	\begin{summary}
		Related statements to support an assertion
	\end{summary}
		
	\par Together, an argument is a combination of assertions, declarative statements, and a conclusion.
\end{definition}


\begin{definition}{Proposition}
	\begin{summary}
		Informational content of a statement or assertion that can be judged true or false.
	\end{summary}
	\par A good proposition has 5 characteristics:
		\begin{itemize}
			\item It is NOT matters of verifiable fact, or matters of taste, aka they should not have any facts in it at all.
			\item It should not be a question, aka "Doggos are the best" or "Doggos are the worst", not "Are doggos the best?".
			\item It should not introduce bias, aka "The inadequate care of city parks must be improved". You'd need to verify if it is inadequate or not, and if it is, there would be no debate. Instead, the debate should discuss whether it is inadequate, and further action from there.
			\item Unambiguous. "Students with an average grade of B get a scholarship to 'Meme Science'". The question is what the grade average actually is, whether it's in relevant topics or overall.
			\item Singular. One idea at a time.
		\end{itemize}
		
\end{definition}

\begin{definition}{Major \& Minor Propositions}
	\begin{summary}
		Created when any normal arguable proposition is created
	\end{summary}
	
	\par Identify the minor propositions to each major one in order to support the argument. Major: "Captial Punishment should be abolished". Minor: "Captial Punishment is not a deterrent". If you debate these minor propositions, you need to have an argument for it individually.
\end{definition}

\begin{definition}{Deductive Reasoning}
	\begin{summary}
		An argument that flows from a general point to a specific statement.
	\end{summary}
	
	\par It is impossible for the propositions of a deductive argument to be true and the conclusion to be false, as the propositions are more general than the conclusion. 
	\begin{itemize}[label={}]
		\item PREMISE: All puppers are doggos
		\item PREMISE: Bork is a pupper
		\item CONCLUSION: Bork is a doggo
	\end{itemize}
	\par The third item is necessarily true because the argument is definitive proof of the claim.
\end{definition}

\begin{definition}{Inductive Reasoning}
	\begin{summary}
		An argument that flows from facts to more general statements. 
	\end{summary}
	
	\par The premises of an inductive argument are supposed to support the conclusion, such that it is improbable, but not impossible, that the conclusion is false.
	\begin{itemize}[label={}]
		\item PREMISE: Most doggos bork
		\item PREMISE: Bark is a doggo
		\item CONCLUSION: Bark borks.
	\end{itemize}
	\par The third item is probably true but not guaranteed.
\end{definition}

\begin{definition}{Informal Logic}
	\begin{summary}
		Another name for critical thinking	
	\end{summary}
\end{definition}

\begin{definition}{Formal Logic}
	\begin{summary}
		Systems that are constructed to carry out proofs.
	\end{summary}
	
	\par The languages and rules of the reasoning are precisely defined.
\end{definition}

\begin{definition}{Valid Argument}
	\begin{summary}
		An argument is valid if the premises entail their conclusion.
	\end{summary}
	
	\par It is impossible for the propositions to be true and the conclusion be false. The propositions can be false, which doesn't go against the definition of a valid argument, so it is still valid, but it means that the argument is not sound.
\end{definition}

\begin{definition}{Modus Ponens}
	\begin{summary}
		A type of argument
	\end{summary}
	
	\par If P then Q, P therefore Q. Essentially if P is true, Q is also true, but if Q is true, P may not necessarily be true.
\end{definition}

\begin{definition}{Ambiguity}
	\begin{summary}
		One of the 8 fallacies in arguments. Interpret the meaning of an argument differently
	\end{summary}
	
	\par
	\begin{itemize}[label={}]
		\item PREMISE: Computer A's speed is 25MHz
		\item PREMISE: Computer B's speed is 20MHz
		\item CONCLUSION: Computer A will run this program faster than Computer B.
	\end{itemize}
	\par The speed of the computer may not be related to the speed of the program.
\end{definition}

\begin{definition}{Circular Arguments}
	\begin{summary}
		One of the 8 fallacies in arguments. Premises disguise the conclusion
	\end{summary}
	\begin{itemize}[label={}]
		\item PREMISE: The criteria for patentability is A, B and C.
		\item PREMISE: My invention is A, B and C.
		\item CONCLUSION: My invention should be patented.
	\end{itemize}
	\par The second premise is the same as the conclusion. It is not possible for one to be false and another true.
\end{definition}

\begin{definition}{Unwarranted Assumptions}
	\begin{summary}
		One of the 8 fallacies in arguments. Assuming that the whole set of parts automatically has a certain property.
	\end{summary}
	\begin{itemize}[label={}]
		\item PREMISE: Jane is good
		\item PREMISE: John is good
		\item PREMISE: Gordon is good
		\item CONCLUSION: Jane, John and Gordon would make a good team.
	\end{itemize}
	\par Individual properties do not generalise team properties.
\end{definition}

\begin{definition}{Missing Evidence}
	\begin{summary}
		One of the 8 fallacies in arguments. Weak generalisations from biased experiences
	\end{summary}
	\begin{itemize}[label={}]
		\item PREMISE: We need to hire someone.
		\item PREMISE: The last person we hired from X was a fuckwit.
		\item CONCLUSION: We shouldn't hire anyone from X.
	\end{itemize}
	\par Insufficient evidence for the conclusion.
\end{definition}

\begin{definition}{Incorrect Causation}
	\begin{summary}
		One of the 8 fallacies in arguments. Mismatch cause-effect
	\end{summary}
	\begin{itemize}[label={}]
		\item PREMISE: Jane and John had lunch
		\item PREMISE: Jane was selecting the new manager
		\item PREMISE: John became the new manager
		\item CONCLUSION: Jane gives promotions to the people she bangs.
	\end{itemize}
	\par No evidence that the lunch was related to the promotion.
\end{definition}

\begin{definition}{Irrelevant Premises}
	\begin{summary}
		One of the 8 fallacies in arguments. One or more premises not related to the conclusion
	\end{summary}
	\begin{itemize}[label={}]
		\item PREMISE: Winning the contract means we would have to hire new people
		\item PREMISE: It also means lots of overtime work
		\item PREMISE: Office space would be cramped
		\item CONCLUSION: We shouldn't win the contract.
	\end{itemize}		
	\par The second premise doesn't immediately seem to relate to the contract.
\end{definition}

\begin{definition}{Appeal to Emotion, Authority or Beliefs}
	\begin{summary}
		One of the 8 fallacies in arguments. Seeking justification
	\end{summary}
		
	\begin{itemize}[label={}]
		\item PREMISE: Our company is small
		\item PREMISE: Your company is small
		\item PREMISE: Small companies have difficulties in competing
		\item CONCLUSION: You should use our products, not the big companies
	\end{itemize}	
	\par No real argument for the conclusion
\end{definition}

\begin{definition}{Incorrect Deduction}
	\begin{summary}
		One of the 8 fallacies in arguments. Argument doesn't have the correct structure
	\end{summary}
		\begin{itemize}[label={}]

		\item PREMISE: If people were using weak passwords, we should be concerned about security
		\item PREMISE: People aren't using weak passwords
		\item CONCLUSION: We don't need to be worried about security
			\end{itemize}	

	\par Structure of Modus Ponens isn't followed here. If A then B, but Not A doesn't 
\end{definition}

\end{document}
